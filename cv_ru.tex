%-----------------------------------------------------------------------------------------------------------------------------------------------%
%	The MIT License (MIT)
%
%	Copyright (c) 2021 Jitin Nair
%
%	Permission is hereby granted, free of charge, to any person obtaining a copy
%	of this software and associated documentation files (the "Software"), to deal
%	in the Software without restriction, including without limitation the rights
%	to use, copy, modify, merge, publish, distribute, sublicense, and/or sell
%	copies of the Software, and to permit persons to whom the Software is
%	furnished to do so, subject to the following conditions:
%	
%	THE SOFTWARE IS PROVIDED "AS IS", WITHOUT WARRANTY OF ANY KIND, EXPRESS OR
%	IMPLIED, INCLUDING BUT NOT LIMITED TO THE WARRANTIES OF MERCHANTABILITY,
%	FITNESS FOR A PARTICULAR PURPOSE AND NONINFRINGEMENT. IN NO EVENT SHALL THE
%	AUTHORS OR COPYRIGHT HOLDERS BE LIABLE FOR ANY CLAIM, DAMAGES OR OTHER
%	LIABILITY, WHETHER IN AN ACTION OF CONTRACT, TORT OR OTHERWISE, ARISING FROM,
%	OUT OF OR IN CONNECTION WITH THE SOFTWARE OR THE USE OR OTHER DEALINGS IN
%	THE SOFTWARE.
%	
%
%-----------------------------------------------------------------------------------------------------------------------------------------------%

%----------------------------------------------------------------------------------------
%	DOCUMENT DEFINITION
%----------------------------------------------------------------------------------------

% article class because we want to fully customize the page and not use a cv template
\documentclass[a4paper,11pt]{article}

%----------------------------------------------------------------------------------------
%	FONT
%----------------------------------------------------------------------------------------

% % fontspec allows you to use TTF/OTF fonts directly
% \usepackage{fontspec}
% \defaultfontfeatures{Ligatures=TeX}

% % modified for ShareLaTeX use
% \setmainfont[
% SmallCapsFont = Fontin-SmallCaps.otf,
% BoldFont = Fontin-Bold.otf,
% ItalicFont = Fontin-Italic.otf
% ]
% {Fontin.otf}

%----------------------------------------------------------------------------------------
%	PACKAGES
%----------------------------------------------------------------------------------------
\usepackage{url}
\usepackage{parskip} 	
\usepackage[utf8]{inputenc} % allow utf-8 input
\usepackage[T2A]{fontenc}    % use 8-bit T1 fonts
\usepackage[english, russian]{babel}

%other packages for formatting
\RequirePackage{color}
\RequirePackage{graphicx}
\usepackage[usenames,dvipsnames]{xcolor}
\usepackage[scale=0.9]{geometry}

%tabularx environment
\usepackage{tabularx}

%for lists within experience section
\usepackage{enumitem}

% centered version of 'X' col. type
\newcolumntype{C}{>{\centering\arraybackslash}X} 

%to prevent spillover of tabular into next pages
\usepackage{supertabular}
\usepackage{tabularx}
\newlength{\fullcollw}
\setlength{\fullcollw}{0.47\textwidth}

%custom \section
\usepackage{titlesec}				
\usepackage{multicol}
\usepackage{multirow}

%CV Sections inspired by: 
%http://stefano.italians.nl/archives/26
\titleformat{\section}{\large\scshape\raggedright}{}{0em}{}[\titlerule]
\titlespacing{\section}{0pt}{6pt}{6pt}

%for publications
\usepackage[style=authoryear,sorting=ynt, maxbibnames=2]{biblatex}

%Setup hyperref package, and colours for links
\usepackage[unicode, draft=false]{hyperref}
\definecolor{linkcolour}{rgb}{0,0.2,0.6}
\hypersetup{colorlinks,breaklinks,urlcolor=linkcolour,linkcolor=linkcolour}
\addbibresource{citations.bib}
\setlength\bibitemsep{1em}

%for social icons
\usepackage{fontawesome5}
\usepackage{csquotes}

%debug page outer frames
%\usepackage{showframe}

%----------------------------------------------------------------------------------------
%	BEGIN DOCUMENT
%----------------------------------------------------------------------------------------
\begin{document}

% non-numbered pages
\pagestyle{empty} 

%----------------------------------------------------------------------------------------
%	TITLE
%----------------------------------------------------------------------------------------

% \begin{tabularx}{\linewidth}{ @{}X X@{} }
% \huge{Your Name}\vspace{2pt} & \hfill \emoji{incoming-envelope} email@email.com \\
% \raisebox{-0.05\height}\faGithub\ username \ | \
% \raisebox{-0.00\height}\faLinkedin\ username \ | \ \raisebox{-0.05\height}\faGlobe \ mysite.com  & \hfill \emoji{calling} number
% \end{tabularx}

\begin{tabularx}{\linewidth}{@{} C @{}}
\Huge{Дмитрий Крылов} \\[7.5pt]
\href{https://github.com/dm1trykrylov}{\raisebox{-0.05\height}\faGithub\ dm1trykrylov} \ $|$ \ 
%\href{https://linkedin.com/in/username}{\raisebox{-0.05\height}\faLinkedin\ dm1trykrylov} \ $|$ \ 
\href{https://t.me/dm1trykrylov}{\raisebox{-0.05\height}\faGlobe \ tg} \ $|$ \ 
\href{mailto:krylov.de@phystech.edu}{\raisebox{-0.05\height}\faEnvelope \ mail} \ $|$ \ 
\href{tel:+79225131245}{\raisebox{-0.05\height}\faMobile \ +7.922.513.1245} \\
\end{tabularx}

%----------------------------------------------------------------------------------------
% EXPERIENCE SECTIONS
%----------------------------------------------------------------------------------------

%Interests/ Keywords/ Summary
\section{Обо мне}

Я учусь на 2 курсе Физтех-школы Прикладной Математики и Информатики МФТИ.\\ Интересуюсь программированием, в основном бэкенд-разработкой, люблю решать алгоритмические задачи.
Мне нравится разбираться в том, как работают сервисы внутри. Ещё я интересуюсь экономикой и анализом данных.
%----------------------------------------------------------------------------------------
%	EDUCATION
%----------------------------------------------------------------------------------------
\section{Образование}
\begin{tabularx}{\linewidth}{@{}l X@{}}	
%2030 - present & PhD (Subject) at \textbf{University} \hfill \normalsize (GPA: 4.0/4.0) \\

\textbf{2022 - 2026} & \textbf{Московский Физико-Технический Институт}, ФПМИ, Бакалавриат. Направление: Прикладные математика и физика. 
Направленность (профиль): Математическая физика, компьютерные технологии и математическое моделирование в экономике.

\normalsize Средний балл за 1 курс: 4.8/5 \\ 

\end{tabularx}


%----------------------------------------------------------------------------------------
%	COMPETITIONS
%----------------------------------------------------------------------------------------
\section{Соревнования}


\begin{tabularx}{\linewidth}{ @{}l r@{}  }
\href{https://365.finopolis.ru/finodays/}{\textbf{Хакатона FINOdays}} &  \hfill \textbf{Aug 2023}\\[3.75pt]
\multicolumn{2}{@{}X@{}}{
    Вместе с командой участвовал в треке ИИ и ML, кейс Оценка стоимости залогового имущества от Газпромбанка.
Результат: 2 место внутри кейса и 5 на треке из 20 команд.
Мой вклад: сделал сайт на Flask с интерактивными формами и возможностью загрузки табличных данных. \href{https://github.com/SpectrApp/Spectr_App}{Репозиторий проекта}.
}  \\
\end{tabularx}

%Projects
\section{Проекты}

\begin{tabularx}{\linewidth}{ @{}l @{}l r@{} }
    \href{https://github.com/dm1trykrylov/airlines-cutomer-satisfaction}{\textbf{Airlines Customer Satisfaction}} &  $\;\; \mid$ \textit{Python, Matplotlib, Seaborn, Catboost, Streamlit} & \hfill \textbf{Jun 2023}\\[3.75pt]
\multicolumn{2}{@{}X@{}}{
    \begin{itemize}
        \item Провёл разведочный анализ.
        \item Создал \href{https://airlines-customer-satisfaction.streamlit.app/}{сервис} по предсказанию удовлетворённости клиента авиакомпании.
        \item Проект представлен на интенсиве \enquote{Разработка ML-сервиса: от идеи к прототипу} от ФКН ВШЭ. 
        \href{https://drive.google.com/file/d/12kDpbseGoJDyz6SbeZbJjDec-tK012zT/view?usp=sharing}{Сертификат участника}
    \end{itemize}}  \\
    
\end{tabularx}

\begin{tabularx}{\linewidth}{ @{}l @{}l r@{} }
    \href{https://github.com/dm1trykrylov/super-blog}{\textbf{Микроблог}} &  $\;\; \mid$ \textit{Python, Flask, Bootstrap, SQLAlchemy, SQLite} & \hfill \textbf{Dec 2022}\\[3.75pt]
\multicolumn{2}{@{}X@{}}{
    \begin{itemize}
        \item Разработал веб-приложение для блога с регистрацией и авторизацией пользователей, а также новостной лентой.
        \item Задеплоил на \href{https://citadel.pythonanuwhere.com}{pythonanywhere}.
    \end{itemize}}  \\
\end{tabularx}

\begin{tabularx}{\linewidth}{ @{}l @{}l r@{} }
    \href{https://github.com/dm1trykrylov/Sadovod}{\textbf{Приложение Sadovod}} &  $\;\; \mid$ \textit{C\#, Xamarin} & \hfill \textbf{Jul 2020}\\[3.75pt]
\multicolumn{2}{@{}X@{}}{
    \begin{itemize}
        \item Разработал дизайн приложения.
        \item Реализовал обращение к API OpenWeatherMap для получения данных о погоде.
    \end{itemize}}  \\
\end{tabularx}

%----------------------------------------------------------------------------------------
%	PUBLICATIONS
%----------------------------------------------------------------------------------------
%\section{Publications}
%\begin{refsection}[citations.bib]
%\nocite{*}
%\printbibliography[heading=none]
%\end{refsection}

%----------------------------------------------------------------------------------------
%	SKILLS
%----------------------------------------------------------------------------------------
\section{Навыки}
\begin{tabularx}{\linewidth}{@{}l X@{}}
\textbf{Languages:} & \normalsize{C++, Python, C\#, JavaScript} \\
\textbf{Frameworks:} &  \normalsize{Flask, Xamarin, Node.js}\\
\textbf{Developer Tools:}  &  \normalsize{Git, Google Cloud Platform, VS Code, Visual Studio}\\  
\textbf{Libraries:}  &  \normalsize{pandas, numPy, Matplotlib, scikit-learn}\\  
%\textbf{Other:} & \normalsize{ Russian (native), English (Upper-Intermediate)} \\
\end{tabularx}

%Certification
\section{Курсы и Сертификаты}


\begin{tabularx}{\linewidth}{ @{}l r@{} }

\href{https://www.freecodecamp.org/certification/dekrylov/back-end-development-and-apis}{\textbf{Back End Development and APIs}} & \hfill \textbf{Jun 2022} \\[3.75pt]
\multicolumn{2}{@{}X@{}}{
    Certificate issued by freecodecamp.org after successfull completion of Back End Development and APIs course. Tech Stack: JavaScript, Node.js, Express.js.
}  \\
\end{tabularx}


\vfill
\center{\footnotesize Last updated: \today}

\end{document}
