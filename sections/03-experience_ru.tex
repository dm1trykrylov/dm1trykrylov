\begin{tabularx}{\linewidth}{ @{}l @{}l r@{} }
  \href{https://github.com/dm1trykrylov/airlines-cutomer-satisfaction}{\textbf{Airlines Customer Satisfaction}} &  $\;\; \mid$ \textit{Python, Matplotlib, Seaborn, Catboost, Streamlit} & \hfill \textbf{Jun 2023}\\[3.75pt]
\multicolumn{2}{@{}X@{}}{
  \begin{itemize}
      \item Провёл разведочный анализ.
      \item Создал \href{https://airlines-customer-satisfaction.streamlit.app/}{сервис} по предсказанию удовлетворённости клиента авиакомпании.
      \item Проект представлен на интенсиве \enquote{Разработка ML-сервиса: от идеи к прототипу} от ФКН ВШЭ. 
      \href{https://drive.google.com/file/d/12kDpbseGoJDyz6SbeZbJjDec-tK012zT/view?usp=sharing}{Сертификат участника}
  \end{itemize}}  \\
  
\end{tabularx}

\begin{tabularx}{\linewidth}{ @{}l @{}l r@{} }
  \href{https://github.com/dm1trykrylov/super-blog}{\textbf{Микроблог}} &  $\;\; \mid$ \textit{Python, Flask, Bootstrap, SQLAlchemy, SQLite} & \hfill \textbf{Dec 2022}\\[3.75pt]
\multicolumn{2}{@{}X@{}}{
  \begin{itemize}
      \item Разработал веб-приложение для блога с регистрацией и авторизацией пользователей, а также новостной лентой.
      \item Задеплоил на \href{https://citadel.pythonanuwhere.com}{pythonanywhere}.
  \end{itemize}}  \\
\end{tabularx}

\begin{tabularx}{\linewidth}{ @{}l @{}l r@{} }
  \href{https://github.com/dm1trykrylov/Sadovod}{\textbf{Приложение Sadovod}} &  $\;\; \mid$ \textit{C\#, Xamarin} & \hfill \textbf{Jul 2020}\\[3.75pt]
\multicolumn{2}{@{}X@{}}{
  \begin{itemize}
      \item Разработал дизайн приложения.
      \item Реализовал обращение к API OpenWeatherMap для получения данных о погоде.
  \end{itemize}}  \\
\end{tabularx}